% !TEX root = ../main.tex
%
% ========
\chapter{Scripts for correlation function calculations}
\label{a:a}
% ========
  %
  % ========
  \section{Events generation}
  % ========
    In order to perform analysis with sufficient statistics, a large amount of generated events was required.
    To handle this task of generation large amount of data, a computer cluster at Faculty of Physics at Warsaw University of Technology was used.
    This cluster consists of 20 nodes with the following hardware configuration: \mbox{Intel\textregistered} \mbox{Core\texttrademark~2 Quad} CPU \mbox{Q6600 @ 2.40GHz,} \mbox{8GB RAM} with Scientific~Linux~5.8.
    The communication between nodes is realized by the TORQUE Resource Manager~\cite{torque}.
    To control process of launching multiple event generators and collecting the data, the following scripts were written using \verb|Bash| scripting language:
    \begin{description}
      \item[skynet.sh] This is a script in a form of a batch job for TORQUE.
        It simply launches multiple \verb|THERMINATOR| processes in the same working directory with the separate output directory for each job.
        This solution has two advantages: saves space and computation time.
        A single freeze-out hypersurface file has size about 230~MB and when running 20 instances of generator this approach allows to avoid time- and space-consuming copying of the whole \verb|THERMINATOR| directory before running the application.
        The second advantage is a sharing of files containing information about particles' multiplicities and maximum integrands between generator processes (more detailed description is in Section~\ref{sec:events-generation}).
        One can simply execute this batch job using the following command (an example usage):
        \\\verb|qsub -q long -t 0-19 skynet.sh -v dir=th_5.7,events=6000|
        It adds 20 event generators (with task ids from 0 to 19) to the queue, sets the \verb|THERMINATOR| directory as \verb|th_5.7| and sets a number of simulated events to 6000 for each process.
        One has to execute this command in the directory one level higher than \verb|th_5.7| directory.
      \item[merge\_events.sh] After the generation process, one has to merge calculated events into one directory.
        This task requires renaming of a large number of \verb|THERMINATOR| event files.
        Each event generator job produces files named with a certain pattern, starting from event000.root with increasing number.
        In order to move the event files and preserve continuity in the numbering, a simple script was written.
        An example of usage:
        \\\verb&find /data/source -iname "event*.root" -type f \ &
        \\\verb& | merge_events.sh&\\
        This command will find all the event files in the directory /data/source, move and rename those files accordingly to the numeration of events in the current working directory.
    \end{description}
    Sources of these two scripts are available on-line at \url{https://github.com/carbolymer/msc/tree/master/alix}.
  %
  % ========
  \section{Calculations of experimental-like correlation functions}
  % ========
    Correlation functions used in this analysis were calculated using \verb|tpi| software written by Adam Kisiel and was designed for reading event files from \verb|THERMINATOR|.
    It uses \verb|ROOT| library for calculations and storage of the data.
    The application provides functionality of calculation of one-dimensional correlation function in PRF, three-dimensional one in LCMS and its spherical harmonics decomposition (see Section~\ref{sec:sh}).
    The exact numerical procedure of computation of a correlation function is presented in Section~\ref{sec:exp-approach}.
    \verb|tpi| allows to perform calculations with the following options:
    \begin{itemize}
      \item Pair type - there are pion-pion, kaon-kaon, proton-proton and many more pairs available (including ones consisting of non-identical particles)
      \item Multiple $k_T$ subranges from 0.21 to 1.2 GeV/c
      \item Possibility to include Coulomb interaction
      \item Number of events to mix
      \item Maximum freeze-out time
      \item Choice of method of background calculation in correlation function (mixing events or using particles from the same event)
    \end{itemize}
    This program generates results stored in the \textit{*.root} files in a form of histograms.
    Output file contains numerators, denominators and correlation functions from one-dimensional and three-dimensional analysis.
    Moreover, the spherical harmonics series coefficients up to \textit{l}~=~3 with signal and background histograms are stored.
