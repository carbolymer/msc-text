% !TEX root = ../main.tex
\chapter*{Introduction}
\addcontentsline{toc}{chapter}{Introduction}
Many people were trying to discover what was in the beginning of the Universe which we observe today.
Through the years, many theories were formulated to describe its origin and behaviour.
Among them is one model, which provides a comprehensive explanation for a broad range of phenomena, including the cosmic microwave background, abundance of the light elements and Hubble's law.
This model is called The Big Bang theory and has been born in 1927 on the basis of the principles proposed by the Belgian priest and scientist Georges Lema{\^i}tre.
Using this model and known laws of physics one can calculate the characteristics of the Universe back in time to the extreme densities and temperatures.
However, at some point these calculations fail.
The extrapolation of the expansion of the Universe backwards in time using general relativity yields an infinite density and temperature at a finite time in the past.
This appearance of singularity is a signal of the breakdown of general relativity.
The range of this extrapolation towards singularity is debated - certainly we can go no closer than the end of the \textit{Planck epoch} i.e. $10^{-43}$~s.
At this very first era the temperature of the Universe was so high, that the four fundamental forces: electromagnetism, gravitation, weak nuclear interaction and strong nuclear interaction were one fundamental force.
Between $10^{-43}$~s and $10^{-36}$~s of a lifetime of the Universe, there is a \textit{grand unification epoch}, at which forces are starting to separate from each other.
The \textit{electroweak epoch} lasted from $10^{-36}$~s to $10^{-12}$~s, when the strong force separated from the electroweak force.
After this epoch, there was the \textit{quark epoch} in which the Universe was a dense ``soup'' of quarks.
During this stage the fundamental forces of gravitation, electromagnetism, strong and weak interactions had taken their present forms. 
The temperature at this moment was still too high to allow quarks to bind together and form hadrons.
At the end of quark era, there was a big freeze-out - when the average energy of particle interactions had fallen below the binding energy of hadrons.
This era, in which quarks became confined into hadrons, is known as the \textit{hadron epoch}.
At this moment the matter had started forming nuclei and atoms, which we observe today.

Here arises the question: how can we study the very beginning of the Universe?
To do this, one should recreate in a laboratory appropriate conditions i.e. such large density and high temperature.
Today, this is achievable through sophisticated machines like particle accelerators.
Sufficiently high energies are available at the Large Hadron Collider at CERN, Geneva and Relativistic Heavy Ion Collider at Brookhaven National Laboratory in Upton, New York.
In the particle accelerators the heavy ions after being accelerated to near the speed of light are collided in order to generate extremely dense and hot phase of matter and recreate the quark-gluon plasma.
The plasma is believed to behave like an almost ideal fluid, which can be described by the laws of relativistic hydrodynamics.

This thesis is providing predictions for collective behaviour of the quark-gluon plasma coming from the hydrodynamic equations.
Experimental-like analysis was performed for the high energy Pb-Pb collisions generated with the \verb|THERMINATOR| model.

The 1st chapter is an introduction to the theory of heavy ion collisions.
It contains the brief description of the Standard Model and Quantum Chromodynamics.
Moreover the quark-gluon plasma and its signatures are also characterized.

In the 2nd chapter, the relativistic hydrodynamic framework and the \verb|THERMINATOR| model used to perform the simulations of collisions are described.

The 3rd chapter covers the particle interferometry method used in this work.
Predictions coming from the hydrodynamics are also presented here.
An algorithm of building experimental correlation functions is described as well.

In the 4th chapter, an interpretation of the results for two-particle femtoscopy for different pairs of particles is presented.
Moreover, the quantitative analysis of the calculated femtoscopic radii as well as the appearance of transverse mass scaling is discussed.

In the Appendices, the detailed description of the tools developed by the author and used in this work is given.
Utilities for managing the event generation process and plotting of correlation functions are presented.
Furthermore, the fitting software, its design and usage is described.

The results presented in this work were also published in \cite{galazyn}.
They are coming from the same analysis performed by the author.
This thesis and the article are strictly related and were written simultaneously.