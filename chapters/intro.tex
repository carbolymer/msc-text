% !TEX root = ../main.tex
\chapter*{Introduction}
\addcontentsline{toc}{chapter}{Introduction}
Many people were trying to discover what was before the universe which we observe today.
Through the ages there were few more or less successful theories describing its origin and behaviour.
Among them is one model, which describes very well a comprehensive explanation for a broad range of phenomena, including the cosmic microwave background, abundance of the light elements and Hubble's law.
This model is called The Big Bang theory and has born in the 1927 and is based on principles proposed by the Belgian priest and scientist Georges Lema{\^i}tre.
Using this model and known laws of physics one can calculate the characteristics of the universe in detail back in time to the extreme densities and temperatures.
However, at some point this calculations fail.
The extrapolation of the expansion of universe backwards in time using general relativity yields an infinite density and temperature at a finite time in the past.
This appearance of singularity is a signal of the breakdown of general relativity.
The range of this extrapolation towards singularity is debated - certainly we can go no closer than the end of Planck epoch i.e. $10^{-43}$~s.
At this time the temperature of the universe was so high, that the four fundamental forces - electromagnetism, gravitation, weak nuclear interaction and strong nuclear interaction - were one fundamental force.
Little is known about the physics at this temperature.
Between $10^{-43}$~s and $10^{-36}$~s is a grand unification epoch, at which forces separate from each other.

(other epochs, qgp supposedly before few ms of the universe, large accelerators allow to study it and gain knowledge about this state)
