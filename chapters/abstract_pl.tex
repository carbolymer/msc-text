% !TEX root = ../main.tex
W tej pracy zaprezentowane są wyniki analizy dwucząstkowych korelacji pędowych dla trzech różnych typów cząstek produkowanych w zderzeniach ciężkich jonów.
Obliczenia zostały wykonane dla danych ze zderzeń ołów-ołów przy energii w centrum masy $\sqrt{s_{NN}}~=~2.76$~TeV wygenerowanych za pomocą modelu \verb|THERMINATOR| przy użyciu (3+1)-wymiarowego modelu hydrodynamicznego uwzględniającego lepkość ośrodka.
Analiza została wykonana dla trzech rodzajów cząstek: pionów, kaonów i protonów dla dziewięciu różnych przedziałów centralności.

Model \verb|THERMINATOR| pozwala na wykonanie statystycznej hadronizacji stabilnych cząstek jak i również niestabilnych rezonansów z danej hiperpowierzchni wymrażania oraz uwzględnienie propagacji i rozpadów tych rezonansów.
Czterowymiarowa hiperpowierzchnia pochodzi z obliczeń przeprowadzonych na podstawie hydrodynamiki relatywistycznej z uwzględnieniem poprawek pochodzących od lepkości.

Interferometria dwucząstkowa pozwala na zbadanie charakterystyk czasowo-przestrzennych źródła cząstek.
Poprzez analizę danych pochodzących z obliczeń modelowych można dokonać sprawdzenia zakresu stosowalności hydrodynamiki do opisu właściwości plazmy kwarkowo-gluonowej.
Ta praca koncentruje się na weryfikacji skalowania promieni femtoskopowych z masą poprzeczną przewidywanego przez hydrodynamikę.

Trójwymiarowe funkcje korelacyjne zostały obliczone za pomocą rozkładu w szereg harmonik sferycznych.
To podejście wymaga mniejszej statystyki i pozwala na łatwiejszą wizualizację wyników.
Obliczone funkcje korelacyjne wykazują oczekiwany wzrost korelacji dla niskich różnic pędów dla par pionów i kaonów.
Dla par protonów w tym samym zakresie pędów widoczny jest spadek korelacji.
Widoczny jest wpływ pędu poprzecznego pary oraz centralności na funkcję korelacyjną.
W celu wykonania analizy ilościowej tego wpływu, zostało wykonane dopasowanie formuły analitycznej do obliczonych funkcji korelacyjnych.
Otrzymane w ten sposób promienie femtoskopowe w LCMS i PRF wykazują spadek wraz z wzrostem masy poprzecznej $m_T$.
W celu sprawdzenie skalowania przewidywanego przez hydrodynamikę została dopasowana zależność potęgowa: $\alpha m_T^{-\beta}$.
Promienie obliczone dla pionów, kaonów i protonów zachowują wzajemne skalowanie w LCMS.
W przypadku PRF skalowanie nie jest widoczne.
Aby odzyskać skalowanie w PRF, został zaproponowany przybliżony współczynnik: $\sqrt{ \left. \left( \sqrt{\gamma_T} + 2 \right) \middle/ 3 \right. }$.
Promienie w PRF po podzieleniu przez współczynnik skalowania, są opisywalne przez podaną zależność potęgową, zatem umożliwia on odzyskanie skalowania.
Analiza eksperymentalna jest zazwyczaj wykonywana w PRF (wymaga mniejszej statystyki), zatem ta metoda pozwala na łatwiejszą weryfikację przewidywań hydrodynamiki które są widoczne w LCMS, a nie są w PRF.