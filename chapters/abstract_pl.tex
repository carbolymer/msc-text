% !TEX root = ../main.tex

W tej pracy zaprezentowane są wyniki analizy dwucząstkowych korelacji pędowych dla trzech wybranych typów cząstek produkowanych w zderzeniach ciężkich jonów.
Obliczenia zostały wykonane dla danych ze zderzeń ołów-ołów przy energii w centrum masy $\sqrt{s_{NN}}~=~2.76$~TeV wygenerowanych za pomocą modelu \verb|THERMINATOR| przy użyciu (3+1)-wymiarowego modelu hydrodynamicznego uwzględniającego lepkość ośrodka.
Analiza została wykonana dla pionów, kaonów i protonów dla dziewięciu przedziałów centralności.

Model \verb|THERMINATOR| wykonuje statystyczną hadronizację stabilnych cząstek jak i również niestabilnych rezonansów z danej hiperpowierzchni wymrażania, a następnie uwzględnienia propagację i rozpad tych rezonansów.
Czterowymiarowa hiperpowierzchnia pochodzi z obliczeń przeprowadzonych na podstawie hydrodynamiki relatywistycznej z uwzględnieniem poprawek pochodzących od lepkości.

Interferometria dwucząstkowa pozwala na zbadanie charakterystyk czasowo-przestrzennych źródła cząstek.
Poprzez analizę danych pochodzących z obliczeń modelowych można dokonać sprawdzenia zakresu stosowalności hydrodynamiki do opisu właściwości plazmy kwarkowo-gluonowej.
Ta praca koncentruje się na weryfikacji skalowania promieni femtoskopowych z masą poprzeczną przewidywanego przez hydrodynamikę.

Wyliczone trójwymiarowe funkcje korelacyjne zostały rozłożone w szereg harmonik sferycznych.
To podejście wymaga mniejszej statystyki i pozwala na łatwiejszą wizualizację wyników.
Obliczone funkcje wykazują oczekiwany wzrost korelacji dla niskich różnic pędów dla par pionów i kaonów.
Z kolei dla par protonów w tym samym zakresie pędów można zauważyć wyraźny spadek korelacji.
Przy tym, we wszystkich przypadkach zderzeń jest widoczny wpływ pędu poprzecznego pary oraz centralności na funkcję korelacyjną.
W celu wykonania analizy ilościowej tego wpływu, zostało wykonane dopasowanie formuły analitycznej do obliczonych funkcji korelacyjnych.
Otrzymane w ten sposób promienie femtoskopowe w LCMS i PRF wykazują spadek wraz z wzrostem masy poprzecznej $m_T$.
W celu sprawdzenie skalowania przewidywanego przez hydrodynamikę została dopasowana zależność potęgowa: $\alpha m_T^{-\beta}$.
Promienie obliczone dla pionów, kaonów i protonów zachowują wzajemne skalowanie w LCMS.
W przypadku PRF skalowanie nie jest widoczne, więc aby je odzyskać został zaproponowany przybliżony współczynnik skalowania w postaci: $\sqrt{ \left. \left( \sqrt{\gamma_T} + 2 \right) \middle/ 3 \right. }$.
Promienie w PRF po podzieleniu przez tą wartość, dają się opisać oczekiwaną zależnością potęgową.

Analiza eksperymentalna jest zazwyczaj wykonywana w PRF, bowiem wymaga w tym układzie odniesienia mniejszej statystyki.
Zatem metoda odzyskania skalowania pozwala na łatwiejszą, w porównaniu z LCMS, weryfikację przewidywań hydrodynamiki, które nie są widoczne w PRF.