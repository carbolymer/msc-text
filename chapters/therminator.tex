% !TEX root = ../main.tex
\chapter{Therminator model}
  \verb|THERMINATOR|~\cite{therminator} is a Monte Carlo event generator designed to investigate the particle production in the relativistic heavy ion collisions.
  The functionality of the code includes a generation of the stable particles and unstable resonances at the chosen hypersurface model.
  It performs the statistical hadronization which is followed by space-time evolution of particles and the decay of resonances.
  The key element of this method is an inclusion of a complete list of hadronic resonances.
  The second version of \verb|THERMINATOR|~\cite{therminator2} comes with a posibility to incorporate any shape of freeze-out hypersurface and the expansion velocity field, especially those generated externally with various hydrodynamic codes.
  %
  % ========
  \section{Statistical hadronization}
  % ========
    Statistical description of heavy ion collision has been successfully used to describe quantitatively \textit{soft} physics, i.e. the regime with the transverse momentum not exceeding 2 GeV.
    The assumption that hadronic matter before rapid expansion reaches equilibrium, leads to good results in particle abundances measured in heavy ion experiments, in particular, at the high energies.
    At the rather high temperature of the freeze-out $\approx$140-160 MeV, the resonances contribute very significantly to the observables.
    Therefore, the crucial element for the success of the statistical approach is the complete inclusion of hadronic resonances~\cite{therminator}.
    %
    % ========
    \subsection{Cooper-Frye formalism}
    % ========
      % opisac wzor C-F

    %
    % ========
    \section{(3+1)-dimensional viscous hydrodynamics}
    % ========
    Most of the relativistic viscous hydrodynamic calculations are done in \mbox{(2+1)-dimensions}.
    Such simplification assumes boost-invariance of a matter created in a collision.
    Experimental data reveals that no boost-invariant region is formed in the collisions~\cite{chmeson}.
    Hence, for the better description of created system a \mbox{(3+1)-dimensional} model is required.

    In the four dimensional relativistic dynamics one can describe a system using a space-time four-vector $x^\nu=(ct,x,y,z)$, a velocity four-vector $u^\nu=\gamma(c,v_x,v_y,v_z)$ and a energy-momentum tensor $T^{\mu\nu}$.
    The particular components of $T^{\mu\nu}$ have a following meaning:
    \begin{itemize}
      \item $T^{00}$ - an energy density,
      \item $c T^{0\alpha}$ - an energy flux across a surface $x^\alpha$,
      \item $T^{\alpha0}$ - an $\alpha$-momentum flux across a surface  $x^\alpha$ multiplied by $c$,
      \item $T^{\alpha\beta}$ - components of momentum flux density tensor,
    \end{itemize}
    where $\gamma = (1-v^2/c^2)^{-1/2}$ is Lorentz factor and $\alpha,\beta \in \{1,2,3\}$.
    Using $u^\nu$ one can express $T^{\mu\nu}$ as follows~\cite{israel}:
    \begin{equation}
      \label{eq:visc_ten}
      T^{\mu\nu}_0 = (e+p)u^\mu u^\nu - pg^{\mu\nu}
    \end{equation}
    where $e$ is an energy density, $p$ is a pressure and $g^{\mu\nu}$ is an inverse metric tensor:
    \begin{equation}
      g^{\mu\nu} = 
      \begin{bmatrix}
        1 & 0 & 0 & 0 \\
        0 & -1 & 0 & 0 \\
        0 & 0 & -1 & 0 \\
        0 & 0 & 0 & -1
      \end{bmatrix} .
    \end{equation}
    The presented version of energy-momentum tensor (\ref{eq:visc_ten}) can be used to describe dynamics of a perfect fluid.
    To take into account influence of viscosity, one has to apply the following corrections coming from shear $\pi^{\mu\nu}$ and bulk $\Pi$ viscosities~\cite{visc_hydro}:
    \begin{equation}
      T^{\mu\nu} = T_0^{\mu\nu} + \pi^{\mu\nu} + \Pi(g^{\mu\nu}-u^{\mu}u^{\nu}) .
    \end{equation}
    The stress tensor $\pi^{\mu\nu}$ and the bulk viscosity $\Pi$ are solutions of dynamical equations in the second order viscous hydrodynamic framework~\cite{israel}.
    The comparison of hydrodynamics calculations with the experimental results reveal, that the shear viscosity divided by entropy $\eta / s$ has to be small and close to the AdS/CFT estimate $\eta /s$ = 0.08~\cite{visc_hydro,adscft}.
    When using $T^{\mu\nu}$ to describe system evolving close to local thermodynamic equilibrium, relativistic hydrodynamic equations in a form of:
    \begin{equation}
      \partial_{\mu} T^{\mu\nu} = 0
    \end{equation}
    can be used to describe the dynamics of the local energy density, pressure and flow velocity.

    Hydrodynamic calculations are starting from the Glauber \footnote{The Glauber Model is used to calculate ``geometrical'' parameters of a collision like an impact parameter, number of participating nucleons or number of binary collisions.}
    model initial conditions.
    The collective expansion of a fluid ends at the freeze-out hypersurface.
    That surface is usually defined as a constant temperature surface, or equivalently as a cut-off in local energy density.
    The freeze-out is assumed to occur at the temperature $T$ = 140 MeV.
