% !TEX root = ../main.tex
\chapter{Therminator model}
  \verb|THERMINATOR|~\cite{therminator} is a Monte Carlo event generator designed to investigate the particle production in the relativistic heavy ion collisions.
  The functionality of the code includes a generation of the stable particles and unstable resonances at the chosen hypersurface model.
  It performs the statistical hadronization which is followed by space-time evolution of particles and the decay of resonances.
  The key element of this method is an inclusion of a complete list of hadronic resonances, which contribute very significantly to the observables.
  The second version of \verb|THERMINATOR|~\cite{therminator2} comes with a posibility to incorporate any shape of freeze-out hypersurface and the expansion velocity field, especially those generated externally with various hydrodynamic codes.

  %
  % ========
  \section{(3+1)-dimensional viscous hydrodynamics}
  % ========
  Most of the relativistic viscous hydrodynamic calculations are done in \mbox{(2+1)-dimensions}.
  Such simplification assumes boost-invariance of a matter created in a collision.
  Experimental data reveals that no boost-invariant region is formed in the collisions~\cite{chmeson}.
  Hence, for the better description of created system a \mbox{(3+1)-dimensional} model is required.

  In the four dimensional relativistic dynamics one can describe a system using a space-time four-vector $x^\nu=(ct,x,y,z)$, a velocity four-vector $u^\nu=\gamma(c,v_x,v_y,v_z)$ and a energy-momentum tensor $T^{\mu\nu}$.
  The particular components of $T^{\mu\nu}$ have a following meaning:
  \begin{itemize}
    \item $T^{00}$ - an energy density,
    \item $c T^{0\alpha}$ - an energy flux across a surface $x^\alpha$,
    \item $T^{\alpha0}$ - an $\alpha$-momentum flux across a surface  $x^\alpha$ multiplied by $c$,
    \item $T^{\alpha\beta}$ - components of momentum flux density tensor,
  \end{itemize}
  where $\gamma = (1-v^2/c^2)^{-1/2}$ is Lorentz factor and $\alpha,\beta \in \{1,2,3\}$.
  Using $u^\nu$ one can express $T^{\mu\nu}$ as follows~\cite{israel}:
  \begin{equation}
    \label{eq:visc_ten}
    T^{\mu\nu}_0 = (e+p)u^\mu u^\nu - pg^{\mu\nu}
  \end{equation}
  where $e$ is an energy density, $p$ is a pressure and $g^{\mu\nu}$ is an inverse metric tensor:
  \begin{equation}
    g^{\mu\nu} = 
    \begin{bmatrix}
      1 & 0 & 0 & 0 \\
      0 & -1 & 0 & 0 \\
      0 & 0 & -1 & 0 \\
      0 & 0 & 0 & -1
    \end{bmatrix} .
  \end{equation}
  The presented version of energy-momentum tensor (\ref{eq:visc_ten}) can be used to describe dynamics of a perfect fluid.
  To take into account influence of viscosity, one has to apply the following corrections coming from shear $\pi^{\mu\nu}$ and bulk $\Pi$ viscosities~\cite{visc_hydro}:
  \begin{equation}
    T^{\mu\nu} = T_0^{\mu\nu} + \pi^{\mu\nu} + \Pi(g^{\mu\nu}-u^{\mu}u^{\nu}) .
  \end{equation}
  The stress tensor $\pi^{\mu\nu}$ and the bulk viscosity $\Pi$ are solutions of dynamical equations in the second order viscous hydrodynamic framework~\cite{israel}.
  The comparison of hydrodynamics calculations with the experimental results reveal, that the shear viscosity divided by entropy $\eta / s$ has to be small and close to the AdS/CFT estimate $\eta /s$ = 0.08~\cite{visc_hydro,adscft}.
  
  When using $T^{\mu\nu}$ to describe system evolving close to local thermodynamic equilibrium, relativistic hydrodynamic equations in a form of:
  \begin{equation}
    \partial_{\mu} T^{\mu\nu} = 0
  \end{equation}
  can be used to describe the dynamics of the local energy density, pressure and flow velocity.

  Hydrodynamic calculations are starting from the Glauber \footnote{The Glauber Model is used to calculate ``geometrical'' parameters of a collision like an impact parameter, number of participating nucleons or number of binary collisions.}
  model initial conditions.
  The collective expansion of a fluid ends at the freeze-out hypersurface.
  That surface is usually defined as a constant temperature surface, or equivalently as a cut-off in local energy density.
  The freeze-out is assumed to occur at the temperature $T$ = 140 MeV.

  %
  % ========
  \section{Statistical hadronization}
  % ========
    Statistical description of heavy ion collision has been successfully used to describe quantitatively \textit{soft} physics, i.e. the regime with the transverse momentum not exceeding 2 GeV.
    The basic assumption of the statistical approach of evolution of the quark-gluon plasma is that at some point of the space-time evolution of the fireball, the thermal equilibrium is reached.
    When the system is in the thermal equilibrium the local phase-space densities of particles follow the Fermi-Dirac or Bose-Einstein statistical distributions.
    At the end of the plasma expansion, the freeze-out occurs.
    The freeze-out model incorporated in the \verb|THERMINATOR| model assumes, that chemical and thermal freeze-out occur at the same time.
    %
    % ========
    \subsection{Cooper-Frye formalism}
    % ========
      The result of the hydrodynamic calculations is the freeze-out hypersurface~$\Sigma^\mu$.
      A three-dimensional element of the surface is defined as~\cite{therminator2} 
      \begin{equation}
      d\Sigma_\mu = \epsilon_{\mu\alpha\beta\gamma} \frac{\partial x^\alpha}{\partial \alpha} \frac{\partial x^\beta}{\partial \beta} \frac{\partial x^\gamma}{\partial \gamma} d\alpha d\beta d\gamma ,
      \end{equation}
      where $\epsilon_{\mu\alpha\beta\gamma}$ is the Levi-Civita tensor and the variables $\alpha, \beta, \gamma \in \{1,2,3\}$ are used to parametrize the three-dimensional freeze-out hypersurface in the Minkowski four-dimensional space.
      The Levi-Civita tensor is equal to 1 when the indices form an even permutation (eg.~$\epsilon_{0123}$), to -1 when the permutation is odd (e.g. $\epsilon_{2134}$) and has a value of 0 if any index is repeated. Therefore~\cite{therminator2},
      \begin{equation}
        d \Sigma_0  = 
        \begin{vmatrix}
          \frac{\partial x}{\partial \alpha} & \frac{\partial x}{\partial \beta} & \frac{\partial x}{\partial \gamma} \\
          \frac{\partial y}{\partial \alpha} & \frac{\partial y}{\partial \beta} & \frac{\partial y}{\partial \gamma} \\
          \frac{\partial z}{\partial \alpha} & \frac{\partial z}{\partial \beta} & \frac{\partial z}{\partial \gamma} \\
        \end{vmatrix} d \alpha d \beta d \gamma
      \end{equation}
      and the remaining components are obtained by cyclic permutations of \textit{t}, \textit{x}, \textit{y} and~\textit{z}.

      One can obtain the number of hadrons produced on the hypersurface $\Sigma^{\mu}$ from the Cooper-Frye formalism.
      The following integral yields the total number of created particles~\cite{therminator2}:
      \begin{equation}
        \label{eq:cooper-frye}
        N = (2s+1) \int \frac{d^3p}{(2\pi)^3 E_p} \int d\Sigma_{\mu}(x)p^\mu f(x,p) ,
      \end{equation}
      where
      \begin{equation}
        \label{eq:phase-space-dist}
        f(p \cdot u) = \left\{ ex\left[ \frac{p_\mu u^\mu - (B \mu_B + I_3 \mu_{I_3}+S\mu_S + C \mu_C)}{T} \right] \pm 1 \right\}^{-1}
      \end{equation}
      is the phase-space distribution for particles (for stable ones and resonances).
      For the Fermi-Dirac distribution in the \ref{eq:phase-space-dist} there is a plus sign and for Bose-Einstein statistics minus sign respectively.
      The thermodynamic quantities appearing in the $f(\cdot)$ are $T$ - temperature, $\mu_B$ - baryon chemical potential, $\mu_{I_3}$ - isospin chemical potential, $\mu_S$ - strange chemical potential, $\mu_C$ - charmed chemical potential and the $s$ is a spin of a particle.
      One can simply derive from equation \ref{eq:cooper-frye}, the dependence of the momentum density~\cite{cooperfrye}:
      \begin{equation}
        E \frac{d N}{d^3 p} = \int f(x,p)p^\mu d \Sigma_\mu .
      \end{equation}
      The equations presented above are directly used in the \verb|THERMINATOR| to generate the hadrons with the Monte-Carlo method.