% !TEX root = ../main.tex
% An english abstract

This thesis presents results of two-particle momentum correlations analysis for selected types of particles produced in heavy ion collisions.
The studies were carried for the data from lead-lead collisions at the centre of mass energy \mbox{$\sqrt{s_{NN}}=2.76$~TeV} simulated in the \verb|THERMINATOR| model using the (3+1)-dimen-sional hydrodynamic model with viscosity.
Analysis was performed for the three particle types: pions, kaons and protons for the collisions in eight centrality ranges.

The \verb|THERMINATOR| model allows to perform statistical hadronization of stable particles and unstable resonances from a given hypersurface.
It is followed by the resonance propagation and decay phase.
The four-dimensional hypersurface is taken from the calculations performed in relativistic hydrodynamic framework with the viscosity corrections.

One can investigate space-time characteristics of the particle-emitting source through two-particle interferometry.
The experimental-like analysis of the data coming from a model calculations yields the possibility to test the hydrodynamic description of a quark-gluon plasma.
This thesis concentrates on the verification of the prediction of the appearance of scaling of femtoscopic radii with transverse mass.

The three dimensional correlation functions were calculated using spherical harmonics decomposition.
One can use this approach to perform calculations with lower statistics and moreover the visualization of results is much easier.
The calculated correlation functions show expected increase of a correlation for pions and kaons at the low relative momenta of a pair.
For the protons at the same momentum region, the decrease occurs.
Furthermore, the transverse pair momentum and centrality dependence on a correlation function is observed.
In order to perform the quantitative analysis of this influence, the fitting of theoretical formula for correlation function was performed.
The femtoscopic radii calculated in the Longitudinally Co-Moving System and Pair Rest Frame are falling with the transverse mass $m_T$.
To test the scaling predicted from the hydrodynamics, the power-law $\alpha m_T^{-\beta}$ was fitted.
The radii calculated for pions, kaons and protons in the LCMS are following the common scaling.
In the case of the PRF such scaling is not observed.
To recover the scaling in the PRF, the approximate factor is proposed: $\sqrt{ \left. \left( \sqrt{\gamma_T} + 2 \right) \middle/ 3 \right. }$.
The radii in the PRF divided by the proposed scaling factor are falling on the common curve, proving that the scaling can be recovered using the proposed factor.
The experimental analysis is usually performed in the PRF (requires less statistics), hence the method of scaling recovery enables easier testing of the hydrodynamic predictions, also in the PRF.
