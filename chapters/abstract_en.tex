% !TEX root = ../main.tex
% An english abstract
This thesis presents results of two-particle momentum correlations analysis for different kinds of particles produced in heavy ion collisions.
The studies were carried for the data from lead-lead collisions at the centre of mass energy $\sqrt{s_{NN}}~=~2.76$~TeV simulated in the \verb|THERMINATOR| model using the (3+1)-dimensional hydrodynamic model with viscosity.
Analysis was performed for the three particle kinds: pions, kaons and protons for the collisions in eight different centrality ranges.

The \verb|THERMINATOR| model allows to perform statistical hadronization of stable particles and unstable resonances from a given hypersurface which is followed by the resonance propagation and decay phase.
The four-dimensional hypersurface is coming from the calculations performed on a basis of relativistic hydrodynamic framework with the viscosity corrections.

One can investigate space-time characteristics of the particle-emitting source through two-particle interferometry using experimental observables.
The experimental-like analysis of the data coming from a model calculations yields a possibility to test the hydrodynamic description of a quark-gluon plasma.
This thesis concentrates on the verification of the prediction of appearance of femtoscopic radii scaling with the transverse mass.

The three dimensional correlation functions were calculated using spherical harmonics decomposition.
One can use this approach to perform calculations with the less statistics and the visualization of results is much easier.
The calculated correlation functions show expected increase of a correlation for pions and kaons at the low relative momenta of a pair.
For the protons at the same momentum region, the decrease occurs.
The transverse pair momentum and centrality dependence on a correlation function is observed.
In order to perform the quantitative analysis of this influence, the fitting of theoretical formula for correlation function was performed.
The femtoscopic radii calculated in the LCMS and PRF are falling with the transverse mass $m_T$.
To test the scaling predicted from the hydrodynamics, the power law was fitted $\alpha m_T^{-\beta}$.
The radii calculated for pions, kaons and protons in the LCMS are following the common scaling.
In case of the PRF no such scaling is observed.
To recover the scaling in the PRF, the approximate factor to recover scaling is proposed: $\sqrt{ \left. \left( \sqrt{\gamma_T} + 2 \right) \middle/ 3 \right. }$.
The radii in the PRF divided by the proposed scaling factor are falling on the common curve, therefore the scaling can be recovered using the proposed scaling factor.
The experimental analysis is usually performed in the PRF (requires less statistics), hence the method of scaling recovery enables easier testing of the hydrodynamic predictions, which are not visible in the PRF.
