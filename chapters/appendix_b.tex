% !TEX root = ../main.tex
%
% ========
\chapter{Fitting utilities}
\label{a:b}
% ========
  Procedure of fitting analytical formulas to experimental-like correlation functions was performed using custom software written in \verb|C++| and \verb|Bash|.
  This application utilizes \verb|MINUIT|~\cite{minuit} package built in the \verb|ROOT| library.

  The source of fitting software is available on-line at \url{https://github.com/carbolymer/msc/tree/master/fitting}.
  %
  % ========
  \section{Minuit package}
  % ========
    The \verb|MINUIT| is a physics analysis tool for function minimization written in Fortran programming language.
    This tool was designed for statistical analysis and it is working on $\chi^2$ or log-likelihood functions to compute the best-fit parameter values and uncertainties, including correlations between parameters.
    It is implemented in \verb|ROOT| environment as \verb|TMinuit| class, which provides interface to the minimization tool.
    The analysis performed in this work uses \verb|MINUIT| with the Migrad minimization method.
    The Migrad minimizer is the best one embedded in Minuit.
    It is a variable-metric method with inexact line search, a stable metric updating scheme, and checks for positive-definiteness~\cite{minuit}.
  %
  % ========
  \section{Fitting software}
  % ========
    Fitting utility provides tools for extraction of femtoscopic radii from correlation functions for identical particles.
    It provides also a macro for generating plots with these radii as a function of transverse mass and fitting power-law $\alpha m_T^{-\beta}$ to the results.
    %
    % ========
    \subsection{Input parameters}
    % ========
      The application reads the output files from the \verb|tpi| program and extracts from them one-dimensional and three-dimensional correlation functions.
      The latter ones are in a form of spherical harmonics series coefficients.

      One has a possibility to set fit parameters for certain centrality bins, pair types and $k_T$ ranges.
      Configuration files (\textit{*.conf}) are located inside the application's folder in the \verb|data/| directory.
      Files with the names beginning with \textit{fitsh} contain parameters for three-dimensional fits, while \textit{fit1d} prefix indicates settings for one-dimensional ones.
      In principle, names of the parameter files have to follow this pattern: \textit{\{fit type\}.\{centrality\}.\{pair type\}.\{kT range beginning\}.conf}, for example: \verb|fit1d.b3.1.pp.0.6.conf|.
      To set parameters for all fits of the particles of the same kind, one has to create file with the following name: \textit{\{fit type\}.\{pair type\}.conf}. For example \verb|fitsh.kk.conf| contains initial parameters for all fits for kaon pairs.
      Similarly, one can set fit parameters for pions (\textit{pipi}) and protons (\textit{pp}) using corresponding letters in place of \textit{kk} in the name of the file.


      Here is an example parameter file for one-dimensional fit (\textbf{fit1d}):
      \begin{longtable}{p{0.2\linewidth}p{0.2\linewidth}}
        \verb|1.0 L|   & normalization\\
        \verb|1.0 L|   & $\lambda$\\
        \verb|4.0 L|   & $R_{inv}$\\
        \verb|0.0 F|   & not used \\
        \verb|0.0 F|   & not used \\
        \verb|0.0 F|   & not used \\
        \verb|0.0 F|   & not used \\
      \end{longtable}
      The \textit{F} letter after the parameter indicates that it is a fixed value (will not change during fitting process), whereas the \textit{L} parameter tells that this value will be modified.

      An example parameter file for three-dimensional fit (\textbf{fitsh}):
      \begin{longtable}{p{0.3\linewidth}p{0.5\linewidth}}
        \verb|4.5 L 1.2 5.5|      & $R_{out}$ in fm\\
        \verb|4.5 L 1.2 5.5|      & $R_{side}$ in fm\\
        \verb|4.5 L 1.2 6.5|      & $R_{long}$ in fm\\
        \verb|0.70  L 0.2 2.2|    & $\lambda$\\
        \verb|1.14 F 1.14 1.14|   & $C_2^0$ coefficient\\
        \verb|1.25 F 1.25 1.25|   & $C_2^2$ coefficient\\
        \verb|1.0 L 0.8 1.2|      & overall normalization\\
        \verb|0.0 F 0.0 0.0|      & $C_2^0$ normalization\\
        \verb|0.0 F 0.0 0.0|      & $C_2^2$ normalization\\
        \verb|0.25 F 0.25 0.25|   & $q_{beg}$ \\
        \verb|0.25 F 0.25 0.25|   & $q_{slope}$ \\
        \verb|0.0 F 0.0 0.0|      & not used\\
        \verb|0.0 F 0.0 0.0|      & not used\\
        \verb|0.0 F 0.0 0.0|      & not used\\
        \verb|0.0 F 0.0 0.0|      & not used\\
        \verb|0.0 F 0.0 0.0|      & not used\\
        \verb|0.0 F 0.0 0.0|      & not used\\
        \verb|0.0 F 0.0 0.0|      & not used\\
        \verb|0.0 F 0.0 0.0|      & not used\\
        \verb|0.0 F 0.0 0.0|      & not used\\
        \verb|IdLCYlm|            & correlation function numerator name\\
        \verb|0.0075|             & beginning of the fitting range ($q$ in GeV/c)\\
        \verb|0.2|                & end of the fitting range ($q$ in GeV/c)\\
        \verb|0|                  & not used\\
        \verb|0|                  & not used\\
        \verb|0|                  & not used\\
        \verb|0|                  & not used\\
      \end{longtable}
      This file contains extra columns indicating allowed range for value of a fit parameter.
      The first number (the 3rd column) is the minimum and the second one (4th column) is the maximum of this range.

    %
    % ========
    \subsection{Output format}
    % ========
      The application during calculations creates subdirectories for each centrality inside the \verb|data/| directory.
      For each pair type and each of the following variables $R_{inv}$, $R_{out}$, $R_{side}$, $R_{long}$, $\lambda$ and $R_{LCMS}$, the output files \textit{*.out} with four columns are created.
      First column is the beginning of the $k_T$ range, second one is the ending of the range, third column contains result of the fit and the last one stores uncertainty of this value.
      In addition, plots (in the png format) of the correlation functions for each pair type and $k_T$ bin are generated inside subdirectories.

      Files \textit{filelist.\{pair type\}.in} contain list of input \textit{*.root} files with processed correlation functions.
    %
    % ========
    \subsection{Compilation}
    % ========
      This utility requires \verb|ROOT| framework and \verb|libboost-regex-dev| library.
      Compilation can be performed using \verb|make| command inside application's directory.
    %
    % ========
    \subsection{Usage}
    % ========
      %
      % ========
      \subsubsection{Fitting process}
      % ========
        In order to start the fitting process, one should execute the following command:\\
        \verb|./run.sh /path/to/the/tpi/output centrality|\\
        The \verb|/path/to/the/tpi/output| parameter is a location of \verb|tpi| output files and \verb|centrality| is a name of a directory in which the results of fits will be stored.
      %
      % ========
      \subsubsection{Plotting}
      % ========
        In order to plot femtoscopic radii and perform fitting of power law, one has to use the following command:\\
        \verb|make plots|\\
        Plots will be generated in the \verb|output| directory.

