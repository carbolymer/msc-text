% !TEX root = ../main.tex
%
% ========
\chapter{Plotting scripts}
\label{a:c}
% ========
  %
  % ========
  \section{Correlation functions plots}
  % ========
    Plots containing correlation functions were generated using two \verb|ROOT| macros written in \verb|C++|.
    
    The first one, \textbf{cf1DAllCentralities.C} generates two plots with one-dimen-sional correlation functions.
    One of them presents $k_T$ dependence of a correlation function (saved in the \textit{cfvskt.eps} file), while the other one one shows influence of centrality on a correlation function (saved in the \textit{cfvsctr.eps}) file.
    In order to generate plots, one has to set in the line 9 the path to the folder including subdirectories with correlation functions files.
    One can produce eps files, using the following command:\\
    \verb|root -l -b -q cf1DAllCentralities.C|\\
    \textit{cfvskt.eps} and \textit{cfvsctr.eps} files will be generated in the current working directory.

    The second macro, \textbf{cf3DAllCentralities.C} produces plots with spherical harmonics coefficients for pions (\textit{cf3dpi.eps}), kaons (\textit{cf3dk.eps}) and protons (\textit{cf3dp.eps}).
    This script has similar structure to the previous one.
    Like in the previous case, one also has to set the proper path in the line 18.
    To execute this macro, one can use this command:\\
    \verb|root -l -b -q cf3DAllCentralities.C|\\
    As a result, the three files with the output plots will be generated in the current working directory.

    The sources of these plotting macros are available on-line at \url{https://github.com/carbolymer/msc/tree/develop/fitting/macros}.
  %
  % ========
  \section{Plots with femtoscopic radii}
  % ========
    Femtoscopic radii plots can be generated using fitting tool described in Appendix~\ref{a:b}.
    To do so, one has to invoke the following command:\\
    \verb|make plots|\\
    This command executes \verb|ROOT| macro \textit{src/plotter.C} which generates all plots for every centrality automatically.
    This script for every pair type, $k_T$ and centrality bin reads the femtoscopic radii and calculates the transverse mass.
    Afterwards, to all of the femtoscopic radii as a function of transverse mass, the following formula is fitted: $R_x=\alpha m_T^{-\beta}$.
    It should be noted, that the fitting is performed to the radii of pions, kaons and protons together.
    As a result of this process, plot with the femtoscopic radii in the outward, sideward, and longitudinal direction as well as overall radii is generated for every centrality.
    Moreover, the comparison between different centralities in PRF divided by the scaling factor (see Section~\ref{sec:pi-scaling}) and the $R_{LCMS}$ are also plotted.

