% !TEX root = ../main.tex
%
% ========
\chapter{Plotting scripts}
\label{a:c}
% ========
  %
  % ========
  \section{Correlation functions plots}
  % ========
    Plots containing correlation functions were generated using two \verb|ROOT| macros written in \verb|C++|.
    
    The first one, \textbf{cf1DAllCentralities.C} generates two plots with one-dimen-sional correlation functions.
    First plot presents $k_T$ dependence of a correlation function and is saved in the \textit{cfvskt.eps} file, and the second one shows influence of centrality on a correlation function (saved in the \textit{cfvsctr.eps}) file.
    In order to generate plots one has to set in the line 9, the path to the folder which contains sub-directories with files storing correlation functions.
    One can produce eps files, using the following command:\\
    \verb|root -l -b -q cf1DAllCentralities.C|\\
    \textit{cfvskt.eps} and \textit{cfvsctr.eps} files will be generated in the current working directory.

    The second macro, \textbf{cf3DAllCentralities.C} produces plots with spherical harmonics coefficients for pions (\textit{cf3dpi.eps}), kaons (\textit{cf3dk.eps}) and protons (\textit{cf3dp.eps}).
    This script has similar structure to the previous one.
    In order one has to set the proper path in the line 18 also.
    To execute this macro, one can use the following command:\\
    \verb|root -l -b -q cf3DAllCentralities.C|\\
    As a result, the three files with the output plots will be generated in the current working directory.

    The sources of these plotting macros are available on-line on \url{https://github.com/carbolymer/msc/tree/develop/fitting/macros}.
  %
  % ========
  \section{Plots with femtoscopic radii}
  % ========
    Femtoscopic radii plots can be generated using fitting tool described in Appendix~\ref{a:b}.
    To do so, one has to invoke the following command:\\
    \verb|make plots|\\
    This command invokes \verb|ROOT| macro \textit{src/plotter.C} which generates all plots for the every centrality automatically.
    This macro reads the femtoscopic radii for each pair type, $k_T$ centrality bin and calculates the transverse mass for each particle kind.
    Afterwards, to all of the femtoscopic radii as a function of transverse mass, the following formula is fitted: $R_x=\alpha m_T^{-\beta}$.
    It should be noted, that the fitting is performed to the radii of pions, kaons and protons together.
    As a result of the fit, for the each centrality, plot with femtoscopic radii in the outward, sideward, and longitudinal direction as well as overall radii is generated.
    Moreover, the comparison between different centralities in PRF divided by the scaling factor (see Section~\ref{sec:pi-scaling}) and in the $R_{LCMS}$ is also plotted.

