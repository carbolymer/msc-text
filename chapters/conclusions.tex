% !TEX root = ../main.tex
\chapter*{Conclusions}
\addcontentsline{toc}{chapter}{Conclusions}
  This thesis presents the results of the two-particle femtoscopy for different particle kinds produced in Pb-Pb collisions at the centre of mass energy \mbox[$\sqrt{s_{NN}} = 2.76$~TeV].
  The analysed data was generated by the \verb|THERMINATOR| model using hypersurfaces from (3+1)-dimensional hydrodynamic calculations.

  The momentum correlations were studied for three different types of particle pairs: pions, kaons and protons.
  The data was analyzed for eight different sets of initial conditions corresponding to the following centrality ranges: 0-5\%, 0-10\%, 10-20\%, 20-30\%, 30-40\%, 40-50\%, 50-60\% and 60-70\%.
  The correlation functions were calculated for nine $k_T$ bins from 0.1~Gev/c to 1.2~GeV/c.
  The calculations were performed using spherical harmonics decomposition of a three-dimensional correlation function.
  Using this approach, one can obtain full three-dimensional information about the source size using only the three coefficients: $\Re C^0_0$, $\Re C^0_2$ and $\Re C^2_2$.
  To perform further quantitative analysis, the femtoscopic radii were extracted through fitting procedure.

  The calculated correlation functions show expected increase of a correlation at low relative momenta in the case of identical bosons (pions and kaons) and decrease for identical fermions (protons).
  This effect is especially visible in the first spherical harmonic coefficient $\Re C^0_0$.
  The other two components $\Re C^0_2$ and $\Re C^2_2$ are non-vanishing and provide information about the ratios of radii in the outward, sideward and longitudinal directions.

  An increase of width of a correlation function with the peripherality of a collision and also with the pair transverse momentum $k_T$ is observed for pions, kaons and protons.
  This increase of femtoscopic radii (proportional to the inverse of width) with $k_T$ is related with the $m_T$ scaling coming from the hydrodynamics.

  Hydrodynamic equations are predicting appearance of the common scaling of femtoscopic radii for different kinds of particles with $m_T^{-0.5}$ in LCMS.
  In the results of this work, a common scaling for different particle types is observed in LCMS in the outward, sideward and longitudinal directions.
  The direction-averaged radius $R_{LCMS}$ also shows this power-law behaviour.
  The fitting of a power-law $\alpha m_T^{-\beta}$ to the femtoscopic radii yielded the information that the $\beta$ exponent for the outward and sideward directions is of the order of 0.5, which is consistent with the hydrodynamic predictions.
  For the longitudinal direction, the $\beta$ is bigger (>0.7) than in the other ones, which is an indication of a strong transverse flow.
  Femtoscopic radii in LCMS are following the power-law scaling with the accuracy <~5\% for pions and kaons, and <~10\% for protons.
  
  In the case of the one-dimensional radii $R_{inv}$ calculated in PRF, no common scaling is observed.
  This is a consequence of a transition from LCMS to PRF, which causes the growth of radius in the outward direction and breaks the scaling for different particles.
  However, one can try to correct the influence of the $R_{out}$ growth with an approximate factor $\sqrt{ \left. \left( \sqrt{\gamma_T} + 2 \right) \middle/ 3 \right. }$.
  After the division of the $R_{inv}$ by the proposed factor, the scaling is restored with an accuracy <~10\%.
  In this way, the experimentally simpler measure of the one-dimensional radii can be used as a probe for the hydrodynamic collectivity.

  The \verb|THERMINATOR| model includes hydrodynamic expansion, statistical hadronization, resonance propagation and decay afterwards.
  The $m_T$ scaling is predicted from the pure hydrodynamic calculations.
  However, this study shows that influence of the resonances on this scaling is less than 10\%.